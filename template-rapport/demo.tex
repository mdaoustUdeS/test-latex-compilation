\section{Exemple de titre de section}

\todo[inline]{Exemple de description du contenu du paragraphe}
Write actual content here.
You can split long line.
Like that.

You can also create new paragraph by adding a blank line.

\subsection{Exemples de TODO}

\todoGoal{Exemple de description du contenu du paragraphe}
\todoDoing{Changer la ligne \texttt{\textbackslash todoGoal} en \texttt{\textbackslash todoDoing} avant de travailler sur un paragraph.}
\todoDone{Écrivez en \texttt{\textbackslash todoDone} lorsque que vous croyez que le paragraphe est terminé.}
\missingfigure{Inscrivez \texttt{\textbackslash missingfigure} si vous voules insérer une figure}

\section{Ajouter un tableau}

\begin{table}[h!]
    \centering
    \caption{Titre du \textsubscript{super} tableau}
    \begin{tabular}{lllll}
        \hline
        Col1    & Col2    & Col3    & Col4    & Col5\\
        (unité) & (unité) & (unité) & (unité) & (unité)\\
        \hline\hline
        data & data & data & data & data\\
        data & data & data & data & data\\
        data & data & data & data & data\\
        data & data & data & data & data\\
        \hline
    \end{tabular}
    \label{tab:template-Vout}
\end{table}

\section{Ajouter une citation}

On peut citer
une figure (figure \ref{fig:template-example-image}) ou
un tableau \ref{tab:template-Vout}.

\section{Ajouter une image}

\begin{figure}[H]
    \centering
    \includegraphics[width=0.8\linewidth]{example-image-a}
    \caption{Description de l'image.}
    \label{fig:template-example-image}
\end{figure}
